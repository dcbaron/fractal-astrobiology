\documentclass{article}
\usepackage{amsmath, amssymb}
\usepackage{parskip}

\newcommand{\newc}{\newcommand}
\newc{\lrp}[1]{\left(#1\right)}
\newc{\al}[1]{\begin{align*}#1\end{align*}}
\newc{\defin}{\textbf{Definition. }}
\newc{\R}{\mathbb{R}}
\newc{\E}{\mathbb{E}}
\newc{\RR}{$\R$}
\newc{\lrb}[1]{\left[#1\right]}
\newc{\Db}{D_{\mathrm{box}}}
\newc{\db}{d_{\mathrm{box}}}
\DeclareMathOperator{\fe}{fe}
\DeclareMathOperator{\FE}{FE}
\DeclareMathOperator{\FS}{FS}
\newc{\bits}{\ \mathrm{bits}}
\newc{\Ne}{N(\epsilon)}
\newc{\eps}{\epsilon}
\newc{\ent}[1]{H\!\lrp{#1}}
\newc{\entsum}[2]{-\sum_{#1}p(#2)\log p(#2)}


\title{Entropy, Fractals, and Extraterrestrial Life}
\author{A talk by Daniel Baron}
\date{Monday, May 16 at 4 p.m.}


\begin{document}
\maketitle
\section*{\begin{center}Abstract\end{center}}
What is life, and how do we know it when we see it? These are philosophical questions and may not concern us overmuch in our day-to-day lives, but they are of great practical import in the growing field of astrobiology: the search for life on planets beyond our own. Erwin Schr\"odinger famously suggested that life's central distinguishing characteristic is a state of very low ``entropy'' compared to its environment. Astrobiologists Azua-Bustos and Vega-Martinez, inpsired by this idea, have developed a method for detecting the presence or absence of life by analysing the fractal entropy of image data. In this talk, I will explain the basics of Shannon's entropy (a way to quantify the amount of uncertainty or ``surprise'' in an information source or unknown outcome, and cornerstone of communication theory) and of fractals (sets that seem to fill space more completely than their topological dimension would suggest, and that continually reveal more detail as you ``zoom in''). There is a surprising connection between the two! I will then discuss Azua-Bustos' and Vega-Martinez' research in light of these mathematical theories.
\end{document}